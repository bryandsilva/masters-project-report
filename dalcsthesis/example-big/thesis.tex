% This is bigger example of how to use the Dalhousie CS Thesis style.
\documentclass[12pt,glossary]{dalcsthesis}

%Figures
\usepackage{graphicx}
\usepackage{subfigure}


% Algorithms
\usepackage{algorithmic}
\usepackage{algorithm}

%Theorem
\usepackage{nzproof,theorem}
\newtheorem{theorem}{Theorem}
\newtheorem{proposition}{Proposition}
\newtheorem{lemma}{Lemma}
\newtheorem{definition}{Definition}
\newtheorem{observation}{Observation}
\newtheorem{corollary}{Corollary}
\newtheorem{problem}{Problem}

\begin{document}

\mcs  % options are \mcs, \macs, \mec, \mhi, \phd, and \bcshon
\title{How to Write Theses\\
       With Two Line Titles}
\author{I. Bee. Graduate}

\defenceday{1}
\defencemonth{November}
\defenceyear{2006}
\convocation{May}{2007}


% Use multiple \supervisor commands for co-supervisors.
% Use one \reader command for each reader.

\supervisor{Professor Joe Blogs}
\reader{Professor Peter Zappa}
%\reader{Professor General Reference}
\dedicate{To Dalhousie,\\\vspace{12pt}
a great place,\\\vspace{12pt}
to write a long thesis in LaTeX.}

% These next lines can be deleted as you want the appropriate list in
% your thesis

%\nolistoftables
%\nolistoffigures

\frontmatter

\begin{abstract}
\input{abstract.tex}
\end{abstract}

\printglossary

\begin{acknowledgements}
\input{acknowledgements.tex}
\end{acknowledgements}

\mainmatter

\include{1-introduction}
\include{2-doingTheFigures}
\include{3-dancingWithTables}
\include{4-algorithms}
\include{5-proofs}
\include{6-compile}

\bibliographystyle{plain}
\bibliography{thesis}

\appendix

\include{A-longProof}

\end{document}
