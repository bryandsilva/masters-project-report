% simple.tex - simple Master's thesis sample
% $Id: simple.tex 309 2011-01-28 14:46:48Z vlado $
\documentclass[12pt]{dalcsthesis}
% to prepare draft version use option draft:
%\documentclass[12pt,draft]{dalcsthesis}

\begin{document}
\macs  % options are \mcs, \macs, \mec, \mhi, \phd, and \bcshon
\title{The title}
\author{Bryan Thomas D'silva}
\defenceday{1}
\defencemonth{November}
\defenceyear{2017}
\convocation{May}{2018}

% Use multiple \supervisor commands for co-supervisors.
% Use one \reader command for each reader.

\supervisor{D. Prof. Dirk V. Arnold}
\reader{D. Odaprof}
\reader{A. External}

\nolistoftables
\nolistoffigures

\frontmatter

\begin{abstract}
This is a test document.
\end{abstract}

\begin{acknowledgements}
Thanks to all the little people who make me look tall.
\end{acknowledgements}

\mainmatter

\chapter{Introduction}
\chapter{Background}
\section{An Overview of Android Development}
\subsection{About Android}
Android is an open source, Linux based operating system. It is a rich application framework used to develop applications for mobile devices, televisions, cars, etc. Initially developed by the Open Handset Alliance, it is now led by Google. Application development in Android is Java-based, however, on May 17, 2017, support for Kotlin programming language is added. Android has a market share of over 99 percent due to its open source capabilities allowing for more customization and a large developer and community outreach.
\subsection{Application Fundamentals}
\begin{enumerate}
\item Multi-user Linux System: \newline
Each application is considered as a different user of the system. Android assigns a unique Linux ID to every application, shared only between the application and the system. Since every application runs in isolation, it makes it a safe environment. However, applications are allowed to communicate and share data. For example, the Messenger application and the Contacts application share contacts.
\end{enumerate}
\section{Camera 2 API}
\section{RenderScript}
\chapter{Current System}
\section{CheckMarc}
\chapter{Motivation and Objective}
\section{Storyboard}
\section{Capturing Images using an Autocapture algorithm}
\section{Setting the correct exposure}
\section{Comparing the two images}
\chapter{Methodology}
\chapter{Results}
\chapter{Conclusion}


\bibliographystyle{plain}
\bibliography{simple}

\end{document}
